\documentclass[12pt, a4paper]{article}

\usepackage [utf8]{inputenc}
\usepackage [IL2]{fontenc}
\usepackage [czech]{babel}
\usepackage{graphicx}
\usepackage[numbib]{tocbibind}
\usepackage{hyperref}
\graphicspath{{obrazky/}}
\newcommand{\Break}{\State \textbf{break} }

\title{\includegraphics[width=10cm]{FAV_cmyk}

{\huge Semestrální práce z KIV/TI}

\vspace{0.5cm}
{\LARGE Logické řízení - sanitace nádrží}
\vspace{1cm} 

\Large Lukáš Runt (A20B0226P)

\large {\itshape lrunt@students.zcu.cz}

\vspace{0.1cm}
\Large Miroslav Vdoviak (A20B0268P)

\large \itshape{miravdov@students.zcu.cz}
}
\date{\vspace{6cm} \today}

\begin{document}

\begin{titlepage}
\clearpage\maketitle
\thispagestyle{empty}
\end{titlepage}
\tableofcontents \newpage

\section{Zadání}
\includegraphics[width=14cm]{TI.pdf}

\section{Analýza úlohy}

\section{Automatový model}

\subsection{Stavy}
STAV 0 - Systém není v činnosti \newline 
STAV 1 - Tank A se napouští lihem \newline 
STAV 2 - Tanku A se přečerpává čerpadlem \newline 
STAV 3 - Tank A se plní vodou \newline 
STAV 4 - Tank A se proplachuje dokud není ph v normálu \newline 
STAV 5 - Tank A se vypouští \newline 
STAV 6 - Tank B se napouští lihem \newline 
STAV 7 - Tanku B se přečerpává čerpadlem \newline 
STAV 8 - Tank B se plní vodou \newline 
STAV 9 - Tank B se proplachuje dokud není ph v normálu \newline 
STAV 10 - Tank B se vypouští 

\subsection{Snímače}
LA011 - Hladina dosahuje maxima tanku A \newline 
LA010 - Hladina nedosahuje maxima tanku A \newline 
LA021 - Hladina dosahuje minima tanku A \newline 
LA020 - Hladina nedosahuje minima tanku A \newline 
LA031 - Hladina dosahuje maxima tanku B \newline
LA030 - Hladina nedosahuje maxima tanku B \newline 
LA041 - Hladina nedosahuje minima tanku B \newline 
LA040 - Hladina dosahuje minima tanku B

\subsection{Řídící signály}
P0 - Čerpadlo vyplé \newline 
P1 - Čerpadlo zaplé \newline 
Vi0 - Ventil i zavřen \newline 
Vi1 - Ventil i otevřen \newline 
Q0 - Ph nad požadovanou mezí \newline 
Q1 - Ph pod požadovanou mezí 

\subsection{Řízení operátora}
A - Sanitace tanku A \newline 
B - Sanitace tanku B \newline 
Z - Žárovka

\subsection{Přechodový graf}
\begin{figure}
\centering 
\includegraphics[width=14cm]{prechodovyAutomat}
\caption{Přechodový graf automatu}
\end{figure}

\subsection{Chybové stavy} \label{chyba}

\section{Implementace}

\section{Uživatelská příručka}

\subsection{Spuštění programu}
Aplikace se spoští pomocí příkazu v příkazové řádce. Před zadáním příkazu se musíme ujistit, zda se nacházíme ve stejné složce, jako jar soubor, který se chystáme spustit (\texttt{semestralkaTI.jar}). Aplikaci poté spustíme pomocí příkazu: \texttt{java -jar semestralkaTI.jar}. Pro spuštění je předpokladem mít nainstalovanou Javu verze nejméně 11. Odkaz ke stažení Javy 11: \url{ https://www.oracle.com/java/technologies/downloads/#java11}

\begin{figure}[h]
\centering 
\includegraphics{prikladSpusteni}
\caption{Příklad spuštění}
\end{figure}

Pokud se program podaří spustit zobrazí se model sanitarizace tanků \ref{vzhled}.

\begin{figure}
\centering 
\includegraphics[width=10cm]{pospusteni}
\caption{Vzhled aplikace po spuštění}
\label{vzhled}
\end{figure}

\subsection{Ovládání aplikace}
Po spuštění se zobrazí model ve stavu 0. Červená barva znamená logicloku 0, tedy ventil je zavřené, tlačítko není stlačeno, ph není v požadované mezi, čerpadlo nečerpá líh, hladina v tanku není výš než snímač. Zelená barva znamená naopak logickou 1, tedy ventil je otevřený, tlačítko je stlačeno, atd. Žárovka má své barvy a to šedou pokud nesvítí a žlutou pokud svítí. Voda je znázorněna modrou barvou a líh barvou šedou.

Aplikace se ovládá pomocí klávesnice. Uživatel má k dispozici ovládací panel (Spouštění sanitarizace nádrží) a manuální ovládání, které zahrnuje ovládání jednotlivých ventilů a čerpadla. Při implementaci byla snaha o intuitivní ovládání, tedy ventily se ovládájí pomící jejich čísla, ostatní prvky se ovládají pomocí písmenka, kterým je daný prvek pojmenovaný. Podrobný výčet ovládání je uveden níže \ref{ovladani}.

Aplikace počítá s neobvyklým zacházením. Jsou tedy ošetřeny stavy, při kterých by mohlo dojít k chybě. Při chybovém stavu vyskočí na uživatele upozornění a v modelu se rozsvítí žárovka. Výčet chybových stavů lze najít zde: \ref{chyba}

\subsubsection{Výčet ovládacích kláves} \label{ovladani}
A - Spuštění sanitarizace tanku A \newline 
B - Spuštění sanitarizace tanku B \newline 
P - Manuální spuštění čerpadla \newline 
1 - Manuální otevření ventilu 1 \newline 
2 - Manuální otevření ventilu 2 \newline 
3 - Manuální otevření ventilu 3 \newline 
4 - Manuální otevření ventilu 4 \newline 
5 - Manuální otevření ventilu 5 \newline 
6 - Manuální otevření ventilu 6 \newline 
7 - Manuální otevření ventilu 7 \newline 

\section{Závěr}
Celkovou práci hodnotím pozitivně, neboť jsem si vyzkoušel napsat konečný automat. Byl to pro mne nepopsatelný zážitek, který mě studijně obohatil a posunul o krok blíže k praktickým aplikacím teoreticky získaných vědomostí. 

\end{document}